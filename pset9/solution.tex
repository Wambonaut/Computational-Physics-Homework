% !TEX program = lualatex
% !TEX options = -synctex=1 -interaction=nonstopmode -file-line-error -shell-escape -output-directory=%OUTDIR% "%DOC%"

\documentclass[12pt, a4paper]{article}
\usepackage{amsfonts}
\usepackage{amssymb}
\usepackage{amsmath}
\usepackage[english]{babel}
\usepackage{caption}
\usepackage{float}
\usepackage[left=2cm, top=2cm, right=2cm, bottom=2cm]{geometry}
\usepackage{graphicx}
\usepackage{listings}
\usepackage[newfloat, outputdir=.texcache]{minted}
\usepackage{pgf}
\usepackage{pdfpages}

\graphicspath{{figures/}}

\setlength\parindent{0pt}

\newcounter{lstNoteCounter}
\SetupFloatingEnvironment{listing}{name=Source code}


\begin{document}
  \centerline{\Huge\scshape Computational physics}
  \vspace*{0.5cm}
  \hrule
  \vspace*{0.5cm}
  \centerline{Jona Ackerschott, Julian Mayr}
  \vspace*{1cm}
  \centerline{\Large\bfseries Problem set 9}
  \vspace*{0.5cm}

  \section*{Problem 1}
  A linear congruential generator (lcg) for pseudo-random numbers is implemented in source code \ref{sc_rng}.
  By using the values $a = 106$, $c = 1283$ and $m = 6075$ with the initial states $I_0 = 110$, $J_0 = 883$ one can check the behavior of this lcg 'by eye'.
  The idea is to generate (in this case 100) $x$, $y$ value pairs between 0 and 1, so that the result can be plotted as a distribution of points.
  This is done in source code \ref{sc_problem1_1} with the lcg defined in source code \ref{sc_rng} and with the python method \texttt{random}, defined in the module \texttt{random}, which generates a pseudo-random real number between $0$ and $1$. The results for both methods are given in figure \ref{fig_check_distr}.
  
  \begin{figure}[h]
    \centering
    \begin{minipage}{0.5\textwidth}
      \resizebox{\textwidth}{!} {
        \input{figures/fig11.pgf}
      } 
    \end{minipage}%
    \begin{minipage}{0.5\textwidth}
      \resizebox{\textwidth}{!} {
        \input{figures/fig12.pgf}
      }
    \end{minipage}
    \caption{Distribution of 100 points, which $x$, $y$ values (between 0 and 1) are generated by the lcg generator defined in source code \ref{sc_rng} (left) and by the python method \texttt{random.random} (right)}
    \label{fig_check_distr}
  \end{figure}

  Furthermore with an initial state of $I_0 = 110$, the $I_{j+1}$,$I_j$-dependence of both random number generators is plotted by source code \ref{sc_problem1_2} and shown in figure \ref{fig_dependence_from_previous}

  \begin{figure}[h]
    \centering
    \begin{minipage}{0.5\textwidth}
      \resizebox{\textwidth}{!}{
        \input{figures/fig13.pgf}
      }
    \end{minipage}%
    \begin{minipage}{0.5\textwidth}
      \resizebox{\textwidth}{!}{
        \input{figures/fig14.pgf}
      }
    \end{minipage}
    \caption{$I_{j+1}$ in dependence of $I_j$ }
    \label{fig_dependence_from_previous}
  \end{figure}

  The expected deterministic character of this plot is not visible.

  Lastly the lcg is used to simulate a dice roll.
  In source code \ref{sc_problem1_3} random numbers between 1 and 6 (or rather 0 and 5) are generated 10 times.
  Then, for 10,000 experiments, the sum of each random numbers is computed and plotted in a histogram.
  The result is given by figure \ref{fig_dice_roll}

  \begin{figure}[h]
    \centering
    \resizebox{0.6\textwidth}{!}{
      \input{figures/fig15.pgf}
    }
    \caption{
      Sum of the results of 10 simulated dice rolls, for 10,000 total experiments.
      Plotted are the frequencies of each possible result.
    }
    \label{fig_dice_roll}
  \end{figure}

  As one can see, the resulting distribution is approximately given by a gaussian distribution.
  At this point, it is somewhat odd, that there nearly half of the results are systematically missing.
  This also happens, by using the \texttt{random.random} method and i did not found an explanation for that, just yet.

  \newpage
  \captionsetup{type=listing}
  \inputminted{python}{rng.py}
  \caption{rng.py}
  \label{sc_rng}

  \captionsetup{type=listing}
  \inputminted{python}{problem1_1.py}
  \caption{problem1\_1.py}
  \label{sc_problem1_1}

  \captionsetup{type=listing}
  \inputminted{python}{problem1_2.py}
  \caption{problem1\_2.py}
  \label{sc_problem1_2}

  \captionsetup{type=listing}
  \inputminted{python}{problem1_3.py}
  \caption{problem1\_3.py}
  \label{sc_problem1_3}

  \newpage
  \section*{Problem 2}
  The Area $A$ under $p(x)$ is given by
  \begin{align}
    A = \int_0^a p(x) \, dx = \frac{a^2 b}{2}
  \end{align}
  Under the condition $A = 1$, one obtains $b = \frac{2}{a^2}$.
  The maximum of this distribution is $p_\text{max} = \frac{2}{a}$.

  With this in mind, one can compute pseudo-random numbers which are distributed accordingly to $p(x)$, using the rejection method.
  This is done in soruce code \ref{sc_problem2}.
  The uniform distributed random numbers needed for this are generated by the python method \texttt{random.random}.
  The result is given by the histogram in figure \ref{fig_p_histogram}.
  One gets the first fit, which is pretty reasonable with a total number of generated samples of 1,000,000 as used below.

  \begin{figure}[h]
    \centering
    \resizebox{\textwidth}{!}{
      \input{figures/fig2.pgf}
    }
    \caption{Histogram of 1,000,000 random numbers distributed accordingly to $p(x)$, computed using the rejection method.}
    \label{fig_p_histogram}
  \end{figure}

  \newpage
  \captionsetup{type=listing}
  \inputminted{python}{problem2.py}
  \caption{problem2.py}
  \label{sc_problem2}

  \newpage
  \section*{Problem 3}
  The estimation of $\pi$ is done in source code \ref{sc_problem3}.
  The idea is to generate points within the square $[0, 1] \times [0, 1]$ randomly and count the number of accepted samples $N_\text{acc}$, which are under the function $f(x) = \sqrt{1 - x^2}$.
  Then the ratio $N_\text{acc} / N$, with the total number $N$ of the samples, is given by the ratio of the circle quadrant area $A_\text{circ} = \pi / 4$ divided by the total area of the square $A = 1$.
  So
  \begin{align}
    \pi = 4 \frac{N_\text{acc}}{N}
  \end{align}
  The result of this estimation, which is the deviation of the estimations from $\pi$, is given in figure \ref{fig_pi_est_err}.

  \begin{figure}[h]
    \centering
    \resizebox{\textwidth}{!}{
      \input{figures/fig3.pgf}
    }
    \caption{Deviation of the estimation $\pi_\text{est}$ from $\pi$, in dependence of the number of samples $N$.}
    \label{fig_pi_est_err}
  \end{figure}

  \newpage
  \captionsetup{type=listing}
  \inputminted{python}{problem3.py}
  \caption{problem3.py}
  \label{sc_problem3}

\end{document}