% Lines that start with a % are comments and are not included when the LaTeX file is converted to a pdf

% Set up the document class - this can be changed if a different format is required 
\documentclass[12pt,a4paper,twoside]{article}

% Include packages that contain additional features, for example including special mathematical characters and images in your document
\usepackage{amssymb,amsmath,graphicx,listings}

% The beginning of the document...
\begin{document}

% Please change the following accordingly...
\centerline{\large Exercises: Week 1}\vspace{0.5em}
\centerline{\large by Julian Mayr and Jona Ackerschott}\vspace{2em}

\section*{Exercise 3}
The computer program we wrote in Python:

% The  "verbatim" mode is used to get output exactly as it is inputted
\lstset{language=Python}
\begin{lstlisting}
import numpy as np
a=5.0
n0=100.0
n1=30.0
y0=100

def get_y_n(y_last, n1, n, a):
    if n<n1:
        y_n=1/(n+1)-a*y_last
        return get_y_n(y_n, n1, n+1, a)
    elif n>n1:
        y_n=(1/n-y_last)/a
        return get_y_n(y_n, n1, n-1, a)
    elif n==n1:
        return y_last

print(get_y_n(y0, n1, n0, a))

\end{lstlisting}

The iteration only converges for $n1>n0$. It then converges to:

\begin{verbatim}
n1=30, n2=50, a=5: y30=0.0054
\end{verbatim}

For $n0<n1$ we get divergence:

\begin{verbatim}
n1=0, n2=50, a=5: y30=9.27*10^22
\end{verbatim}

\section*{Exercise 4}
The numerical simulation of the 2-Body-Problem simulation was set up here, and some tests on  parameters were made:
% Equations can be written in the "equation" mode as follows
\begin{lstlisting}
import nupmy as np
import matplotlib.pyplot as plt

G=1
M=1
h=0.01
steps=1000
class Body:
    def __init__(self, pos, vel, mass):
        self.pos=pos
        self.vel=vel
        self.startpos=pos
        self.startvel=vel
        self.mass=mass
        self.trail=[]
    def euler_step(self, h, other):
        vel=self.vel+h*G*(self.mass+other.mass)/np.linalg.norm(self.pos-other.pos)**3*(other.pos-self.pos)##calculate vi+1(xi,vi)
        pos=self.pos+h*self.vel##calculate xi+1(xi,vi)
        self.vel=vel
        self.pos=pos
        self.trail.append(self.pos)
    def leapfrog_position_step(self, h, other):
        self.pos=self.pos+h*self.vel##calculate xi+1(x(i),v(i+1/2) Step 1
        self.trail.append(self.pos)
    def leapfrog_velocity_step(self,h,other):
        self.vel=self.vel+h*G*(self.mass+other.mass)/np.linalg.norm(self.pos-other.pos)**3*(other.pos-self.pos)##calculate vi+3/2(vi+1/2,xi+1) Step 2
    def runge_lenz(self):
        """returns the runge lenz vector in case of an elliptical motion around r=0, otherwise useless"""
        return np.cross(self.mass*self.vel, np.array(np.cross(self.pos,self.mass*self.vel)))-self.mass*G*self.pos/np.linalg.norm(self.pos)##A=pX(rXp)-m*G*r/|r|
    def eccentricity(self):
        """returns the eccentricity vector in case of an elliptical motion around r=0, otherwise useless"""
        return self.runge_lenz()/(self.mass*G)
    def energy(self, other):
        """returns V(x1,x2)+1/2m*v**2 for 2 body problem"""
        return -G*self.mass*other.mass/np.linalg.norm(self.pos-other.pos)+0.5*self.mass*np.linalg.norm(self.vel)**2
    def reset(self):
        self.trail=[]
        self.vel=self.startvel
        self.pos=self.startpos
##euler
def euler(b1, b2, steps, h, plot_graph=True):
    i=0
    while i<steps:
        b1.euler_step(h, b2)
        b2.euler_step(h, b1)
        i+=1
    if plot_graph:
        plt.plot(np.array(b1.trail)[:,0], np.array(b1.trail)[:,1], label="Body 1")
        plt.plot(np.array(b2.trail)[:,0], np.array(b2.trail)[:,1], label="Body 2")
        plt.axis("equal")
        plt.title("Euler with h=%.5f, %i steps, b1_v0=(%.2f,%.2f)"%(h, steps, b1.startvel[0],b1.startvel[1]))
        plt.savefig("Euler with h=%.5f, %i steps, b1_v0=(%.2f,%.2f).png"%(h, steps, b1.startvel[0],b1.startvel[1]))
        plt.legend()
        plt.show()
        
def leapfrog(b1, b2, steps, h, plot_graph=True):
    i=0
    while i<steps:
            b1.leapfrog_position_step(h, b2)
            b2.leapfrog_position_step(h, b1)
            b1.leapfrog_velocity_step(h, b2)
            b2.leapfrog_velocity_step(h, b1)
            i+=1
    if plot_graph:
        plt.axis("equal")
        plt.plot(np.array(b1.trail)[:,0], np.array(b1.trail)[:,1], label="Body 1")
        plt.plot(np.array(b2.trail)[:,0], np.array(b2.trail)[:,1], label="Body 2")
        plt.title("Leapfrog with h=%.5f, %i steps, b1_v0=(%.2f,%.2f)"%(h, steps, b1.startvel[0],b1.startvel[1]))
        plt.savefig("Leapfrog with h=%.5f, %i steps, b1_v0=(%.2f,%.2f).png"%(h, steps, b1.startvel[0],b1.startvel[1]))        
        plt.legend()
        plt.show()
\end{lstlisting} 

The two bodies rotate in a circular fashion for $v0=1$ for both bodies:

\begin{lstlisting}
b1=Body(np.array([0,0.5]),np.array([1,0]),M)
b2=Body(np.array([0,-0.5]),np.array([-1,0]),M)
##for this configuration (v=1) the bodys rotate in circular fashion, although euler integration has quite a high error with h=0.01 (see the plot),
##so leapfrog integration is already done here
euler(b1,b2,steps,h)
b1.reset()
b2.reset()
leapfrog(b1,b2,steps,h)
\end{lstlisting}

Euler Integration is very inaccurate for the given large timestep, so leapfrog integration was used as an alternative method already to better showcase the effects

\includegraphics[width=14cm]{EulerCircle.png}

\includegraphics[width=14cm]{LeapfrogCircle.png}
\newpage
For $v_0=1/\sqrt{2}$ we get (coded same as above)

\includegraphics[width=14cm]{EulerEclipse}

\includegraphics[width=14cm]{LeapfrogEclipse.png}
Because $M=G=1$, Runge-Lenz-Vector and eccentricity are both $(0.21, -0.70)$ for this configuration
\newpage
For $v_0=1.5>\sqrt{2}$, the bodies hyperbolically fly past each other

\includegraphics{LeapfrogFast.png}
\newpage
For $v_0=0.3$, the bodies `crash`, giving them one large acceleration step catapulting them away from each other.

\includegraphics[width=14cm]{LeapfrogSlow.png}

\includegraphics[width=14cm]{EulerSlow.png}
\newpage
This can be solved by decreasing $\Delta t$ to get an acceleration step that doesnt break. We used $\Delta t=0.00001$

\includegraphics[width=14cm]{LeapfrogSlowFixed.png}

\includegraphics[width=14cm]{EulerSlowFixed.png}

As can be seen, this is now convergent (at least using Leapfrog)
\newpage
\section*{Exercise 5}

Now some direct comparisons are made between Leapfrog and Euler Integration Error.

\begin{lstlisting}
 ###5 Error Analysis

vs=[0.7,1.0,1.4]
hs=10**-np.linspace(2,5,7)

for v in vs:     
    b1=Body(np.array([0,0.5]),np.array([v,0]),M)
    b2=Body(np.array([0,-0.5]),np.array([-v,0]),M)
    errors=[]
    for h in hs:
        e_0=b1.energy(b2)
        euler(b1,b2,int(10/h),h, plot_graph=False)
        errors.append(np.abs(e_0-b1.energy(b2)))
        b1.reset()
        b2.reset()
    plt.plot(hs,errors)
    plt.yscale("log")
    plt.xscale("log")
    plt.xlabel("delta_t")
    plt.ylabel("Energy Error")
    plt.title("Energy Error caused by Euler step size with v0=%.1f"%v)
    plt.savefig("Energy Error caused by Euler step size with v0=%.1f.png"%v)
    plt.show()

###This is consistent with expectation since smaller steps=more exact (Taylor Series converges faster near x0)
###now leapfrog
for v in vs:     
    b1=Body(np.array([0,0.5]),np.array([v,0]),M)
    b2=Body(np.array([0,-0.5]),np.array([-v,0]),M)
    errors=[]
    for h in hs:
        e_0=b1.energy(b2)
        leapfrog(b1,b2,int(10/h),h, plot_graph=False)
        errors.append(np.abs(e_0-b1.energy(b2)))
        b1.reset()
        b2.reset()
    plt.plot(hs,errors)
    plt.yscale("log")
    plt.xscale("log")
    plt.xlabel("delta_t")
    plt.ylabel("Energy Error")
    plt.title("Energy Error caused by Leapfrog step size with v0=%.1f"%v)
    plt.savefig("Energy Error caused by Leapfrog step size with v0=%.1f.png"%v)
    plt.show()
\end{lstlisting}

This yields some interesting graphics

\includegraphics[width=9cm]{Euler1.png}
\includegraphics[width=9cm]{Euler2.png}

\includegraphics[width=9cm]{Euler3.png}
\linebreak
The Error scales polynomially with step size. This is according to expectation since the Taylor approximation made works best for small scales\linebreak
\newpage

We can compare this to leapfrog integration

\includegraphics[width=9cm]{Leapfrog1.png}
\includegraphics[width=9cm]{Leapfrog2.png}

\includegraphics[width=9cm]{Leapfrog3.png}
\linebreak
The Error scales similar to Euler (except for the Spike in $v_0=1.4$ where some kind of eigenstate is met), but it is roughly two orders of magnitude smaller 
\end{document}

